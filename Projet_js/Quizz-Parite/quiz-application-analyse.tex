\documentclass{article}
\usepackage[utf8]{inputenc}
\usepackage[french]{babel}
\usepackage{listings}
\usepackage{xcolor}
\usepackage{graphicx}
\usepackage{hyperref}

\definecolor{codegreen}{rgb}{0,0.6,0}
\definecolor{codegray}{rgb}{0.5,0.5,0.5}
\definecolor{codepurple}{rgb}{0.58,0,0.82}
\definecolor{backcolour}{rgb}{0.95,0.95,0.92}

\lstdefinestyle{mystyle}{
    backgroundcolor=\color{backcolour},   
    commentstyle=\color{codegreen},
    keywordstyle=\color{magenta},
    numberstyle=\tiny\color{codegray},
    stringstyle=\color{codepurple},
    basicstyle=\ttfamily\footnotesize,
    breakatwhitespace=false,         
    breaklines=true,                 
    captionpos=b,                    
    keepspaces=true,                 
    numbers=left,                    
    numbersep=5pt,                  
    showspaces=false,                
    showstringspaces=false,
    showtabs=false,                  
    tabsize=2
}

\lstset{style=mystyle}

\title{Analyse d'une Application de Quiz}
\author{Documentation Technique}
\date{\today}

\begin{document}

\maketitle

\section{Introduction}
Ce document présente une analyse détaillée d'une application web de quiz. L'application génère aléatoirement 5 questions à partir d'un pool de questions sur les capitales des pays, permet à l'utilisateur d'y répondre, puis évalue les réponses et affiche les résultats. L'application est composée de trois fichiers principaux : un fichier HTML (index.html), un fichier JavaScript (script.js) et un fichier CSS (style.css).

\section{Structure de l'application}
L'application suit une architecture simple basée sur le DOM et les événements JavaScript. Elle permet aux utilisateurs de :
\begin{itemize}
    \item Répondre à 5 questions aléatoires sur les capitales des pays
    \item Soumettre leurs réponses pour évaluation
    \item Voir leur score et les questions correctes/incorrectes
    \item Recommencer le quiz avec un nouveau jeu de questions
\end{itemize}

\section{Analyse du code HTML}
Le fichier HTML fournit la structure de base de l'application :

\begin{lstlisting}[language=HTML]
<!DOCTYPE html>
<html lang="fr">
<head>
    <meta charset="UTF-8">
    <meta name="viewport" content="width=device-width, initial-scale=1.0">
    <title>Quiz JavaScript</title>
    <link rel="stylesheet" href="style.css">
</head>
<body>
    <div class="form-quizz">
        <h1>Quiz JavaScript</h1>
        <form class="form-quizz">
            <!-- Questions will be dynamically inserted here -->
            <div id="questions-container"></div>
            <button type="submit">Valider</button>
        </form>

        <div class="resultats">
            <h2></h2>
            <p class="note"></p>
            <p class="aide"></p>
            <button id="retry-button">Recommencer</button>
        </div>
    </div>

    <script src="script.js"></script>
</body>
</html>
\end{lstlisting}

Les éléments clés sont :
\begin{itemize}
    \item Un titre principal
    \item Un formulaire avec un conteneur vide pour les questions générées dynamiquement
    \item Un bouton de validation pour soumettre les réponses
    \item Une section de résultats avec des éléments pour afficher le score et les messages
    \item Un bouton pour recommencer le quiz
\end{itemize}

La structure est minimaliste car la plupart du contenu est généré dynamiquement par JavaScript.

\section{Analyse du code JavaScript}

\subsection{Initialisation et variables}
Le script commence par sélectionner les éléments DOM nécessaires et définir le pool de questions :

\begin{lstlisting}[language=JavaScript]
const form = document.querySelector(".form-quizz");
const questionsContainer = document.getElementById("questions-container");
const titreResultat = document.querySelector(".resultats h2");
const noteResultat = document.querySelector(".note");
const aideResultat = document.querySelector(".aide");
const retryButton = document.getElementById("retry-button");

// Pool of questions and answers
const questionPool = [
    {
        question: "Quelle est la capitale de la France ?",
        answers: ["Paris", "Lyon", "Marseille"],
        correctAnswer: "a"
    },
    // ... 9 more questions ...
];

let currentQuestions = []; // Stores the current set of questions
let reponses = []; // Stores the correct answers for the current quiz
\end{lstlisting}

Cette partie du code :
\begin{itemize}
    \item Sélectionne les éléments DOM nécessaires pour manipuler l'interface
    \item Définit un tableau d'objets contenant 10 questions sur les capitales des pays
    \item Initialise des variables pour stocker les questions actuelles et les réponses correctes
\end{itemize}

\subsection{Génération des questions}
La fonction \texttt{generateQuestions()} génère aléatoirement 5 questions à partir du pool :

\begin{lstlisting}[language=JavaScript]
function generateQuestions() {
    // Shuffle the question pool and select the first 5 questions
    const shuffledQuestions = questionPool.sort(() => Math.random() - 0.5);
    currentQuestions = shuffledQuestions.slice(0, 5);
    reponses = currentQuestions.map(q => q.correctAnswer);

    // Clear the questions container
    questionsContainer.innerHTML = "";

    // Generate the HTML for the selected questions
    currentQuestions.forEach((question, index) => {
        const questionBlock = document.createElement("div");
        questionBlock.classList.add("question-block");

        const questionText = document.createElement("p");
        questionText.innerText = `${index + 1}. ${question.question}`;
        questionBlock.appendChild(questionText);

        question.answers.forEach((answer, i) => {
            const label = document.createElement("label");
            const input = document.createElement("input");
            input.type = "radio";
            input.name = `q${index + 1}`;
            input.value = String.fromCharCode(97 + i); // a, b, c, etc.
            label.appendChild(input);
            label.appendChild(document.createTextNode(` ${answer}`));
            questionBlock.appendChild(label);
        });

        questionsContainer.appendChild(questionBlock);
    });
}
\end{lstlisting}

Cette fonction :
\begin{itemize}
    \item Mélange le pool de questions en utilisant l'algorithme de Fisher-Yates (via \texttt{sort()} avec une fonction de comparaison aléatoire)
    \item Sélectionne les 5 premières questions du tableau mélangé
    \item Extrait les réponses correctes pour ces questions
    \item Vide le conteneur de questions
    \item Génère dynamiquement le HTML pour chaque question et ses réponses possibles
    \item Ajoute les éléments créés au DOM
\end{itemize}

\subsection{Vérification des réponses}
La fonction \texttt{verifFunc()} vérifie les réponses de l'utilisateur :

\begin{lstlisting}[language=JavaScript]
function verifFunc(tabResultats) {
    const verifTableau = [];
    for (let a = 0; a < 5; a++) {
        verifTableau.push(tabResultats[a] === reponses[a]);
    }
    afficherResultats(verifTableau);
    couleursFonction(verifTableau);
}
\end{lstlisting}

Cette fonction :
\begin{itemize}
    \item Compare les réponses de l'utilisateur avec les réponses correctes
    \item Crée un tableau de booléens indiquant si chaque réponse est correcte
    \item Appelle les fonctions pour afficher les résultats et mettre à jour les couleurs
\end{itemize}

\subsection{Affichage des résultats}
La fonction \texttt{afficherResultats()} affiche le score et les messages appropriés :

\begin{lstlisting}[language=JavaScript]
function afficherResultats(tabCheck) {
    const nbFautes = tabCheck.filter(el => el !== true).length;

    switch (nbFautes) {
        case 0:
            titreResultat.innerText = "Bravo, c'est un sans fautes !";
            aideResultat.innerText = "";
            noteResultat.innerText = "5/5";
            break;
        case 1:
            titreResultat.innerText = "Vous y êtes presque !";
            aideResultat.innerText = "Retentez une autre réponse dans les cases rouges puis validez !";
            noteResultat.innerText = "4/5";
            break;
        // ... cases 2, 3, 4, 5 ...
        default:
            titreResultat.innerText = "Oops, cas inattendu !";
    }
}
\end{lstlisting}

Cette fonction :
\begin{itemize}
    \item Compte le nombre de réponses incorrectes
    \item Utilise une structure \texttt{switch} pour définir les messages appropriés en fonction du score
    \item Met à jour le DOM avec le titre, la note et les conseils
\end{itemize}

\subsection{Mise en évidence des réponses}
La fonction \texttt{couleursFonction()} met en évidence les réponses correctes et incorrectes :

\begin{lstlisting}[language=JavaScript]
function couleursFonction(tabValide) {
    const toutesLesQuestions = document.querySelectorAll(".question-block");
    for (let j = 0; j < tabValide.length; j++) {
        if (tabValide[j] === true) {
            toutesLesQuestions[j].style.background = "lightgreen";
        } else {
            toutesLesQuestions[j].style.background = "#ffbbbb";
            toutesLesQuestions[j].classList.add("echec");
            setTimeout(() => {
                toutesLesQuestions[j].classList.remove("echec");
            }, 500);
        }
    }
}
\end{lstlisting}

Cette fonction :
\begin{itemize}
    \item Sélectionne tous les blocs de questions
    \item Change la couleur de fond en vert pour les réponses correctes
    \item Change la couleur de fond en rouge et ajoute une animation pour les réponses incorrectes
    \item Supprime l'animation après 500 ms
\end{itemize}

\subsection{Gestion des événements}
Le script ajoute des écouteurs d'événements pour la soumission du formulaire et le bouton de recommencement :

\begin{lstlisting}[language=JavaScript]
form.addEventListener("submit", (e) => {
    e.preventDefault(); // Prevent form submission

    const tableauResultats = [];
    for (let i = 1; i <= 5; i++) {
        const selectedAnswer = document.querySelector(`input[name="q${i}"]:checked`);
        tableauResultats.push(selectedAnswer ? selectedAnswer.value : null);
    }

    verifFunc(tableauResultats);
});

retryButton.addEventListener("click", () => {
    generateQuestions(); // Generate new questions
    titreResultat.innerText = "";
    noteResultat.innerText = "";
    aideResultat.innerText = "";
    document.querySelectorAll(".question-block").forEach(block => {
        block.style.background = "white";
    });
});

// Generate questions when the page loads
generateQuestions();
\end{lstlisting}

Cette partie du code :
\begin{itemize}
    \item Ajoute un écouteur pour la soumission du formulaire qui collecte les réponses et les vérifie
    \item Ajoute un écouteur pour le bouton de recommencement qui génère de nouvelles questions et réinitialise l'interface
    \item Appelle \texttt{generateQuestions()} au chargement de la page pour initialiser le quiz
\end{itemize}

\section{Analyse du code CSS}
Le fichier CSS définit l'apparence de l'application :

\begin{lstlisting}[language=CSS]
body {
    font-family: Arial, sans-serif;
    background-color: #f4f4f4;
    margin: 0;
    padding: 20px;
}

h1 {
    text-align: center;
    color: #333;
}

.form-quizz {
    max-width: 600px;
    margin: 0 auto;
    background: #fff;
    padding: 20px;
    border-radius: 8px;
    box-shadow: 0 0 10px rgba(0, 0, 0, 0.1);
}

.question-block {
    margin-bottom: 20px;
}

.question-block p {
    font-weight: bold;
    margin-bottom: 10px;
}

label {
    display: block;
    margin-bottom: 10px;
}

button {
    display: block;
    width: 100%;
    padding: 10px;
    background-color: #28a745;
    color: white;
    border: none;
    border-radius: 5px;
    font-size: 16px;
    cursor: pointer;
}

button:hover {
    background-color: #218838;
}

.resultats {
    margin-top: 20px;
    text-align: center;
}

.resultats h2 {
    color: #333;
}

.note {
    font-size: 24px;
    font-weight: bold;
    color: #28a745;
}

.aide {
    color: #dc3545;
    font-style: italic;
}

.echec {
    animation: shake 0.5s;
}

@keyframes shake {
    0% { transform: translateX(0); }
    25% { transform: translateX(-10px); }
    50% { transform: translateX(10px); }
    75% { transform: translateX(-10px); }
    100% { transform: translateX(0); }
}

#retry-button {
    margin-top: 20px;
    background-color: #007bff;
}

#retry-button:hover {
    background-color: #0056b3;
}
\end{lstlisting}

Les styles principaux incluent :
\begin{itemize}
    \item Une mise en page propre avec une police sans-serif et un arrière-plan clair
    \item Un conteneur central avec des ombres et des coins arrondis
    \item Des styles pour les questions, les étiquettes et les boutons
    \item Des couleurs différentes pour les boutons de validation et de recommencement
    \item Des styles pour les résultats, avec des couleurs différentes pour la note et les conseils
    \item Une animation de secouement pour les réponses incorrectes
\end{itemize}

\section{Flux de l'application}
Le flux de l'application peut être résumé comme suit :

\begin{enumerate}
    \item Au chargement de la page, 5 questions aléatoires sont générées à partir du pool de questions
    \item L'utilisateur sélectionne une réponse pour chaque question
    \item L'utilisateur clique sur le bouton "Valider" pour soumettre ses réponses
    \item L'application vérifie les réponses et affiche le score
    \item Les questions correctes sont mises en évidence en vert, les incorrectes en rouge avec une animation
    \item L'utilisateur peut corriger ses réponses et revalider, ou cliquer sur "Recommencer" pour un nouveau jeu de questions
\end{enumerate}

\section{Aspects techniques importants}

\subsection{Génération aléatoire de questions}
L'application utilise une technique simple mais efficace pour mélanger le tableau de questions :
\begin{lstlisting}[language=JavaScript]
const shuffledQuestions = questionPool.sort(() => Math.random() - 0.5);
\end{lstlisting}

Cette approche, bien que non parfaitement aléatoire d'un point de vue statistique, est suffisante pour cette application.

\subsection{Manipulation dynamique du DOM}
L'application crée dynamiquement les éléments HTML pour les questions et les réponses, ce qui permet de générer un quiz différent à chaque fois.

\subsection{Gestion des réponses}
L'application utilise les attributs \texttt{name} des boutons radio pour regrouper les réponses par question, et les valeurs \texttt{a}, \texttt{b}, \texttt{c} pour identifier les réponses.

\subsection{Feedback visuel}
L'application fournit un feedback visuel immédiat à l'utilisateur en changeant les couleurs des questions et en ajoutant une animation pour les réponses incorrectes.

\section{Limites et améliorations possibles}

\subsection{Persistance des données}
L'application ne persiste pas les scores entre les sessions. Une amélioration possible serait d'utiliser le stockage local (\texttt{localStorage}) pour sauvegarder les meilleurs scores.

\subsection{Catégories de questions}
L'application ne propose qu'un seul type de questions (capitales des pays). Une amélioration possible serait d'ajouter différentes catégories de questions.

\subsection{Niveaux de difficulté}
L'application ne propose qu'un seul niveau de difficulté. Une amélioration possible serait d'ajouter différents niveaux de difficulté.

\subsection{Minuteur}
L'application ne limite pas le temps de réponse. Une amélioration possible serait d'ajouter un minuteur pour chaque question ou pour l'ensemble du quiz.

\subsection{Accessibilité}
L'application pourrait être améliorée en termes d'accessibilité, notamment en ajoutant des attributs ARIA et en s'assurant que tous les éléments interactifs sont accessibles au clavier.

\section{Conclusion}
Cette application de quiz est un exemple bien conçu d'une application web interactive. Elle démontre :
\begin{itemize}
    \item La génération dynamique de contenu HTML
    \item La manipulation du DOM pour créer une interface réactive
    \item L'utilisation d'écouteurs d'événements pour gérer les interactions utilisateur
    \item L'application de styles CSS et d'animations pour améliorer l'expérience utilisateur
    \item Une architecture simple mais efficace pour une application de quiz
\end{itemize}

L'application offre une expérience utilisateur agréable et pourrait servir de base pour une application de quiz plus complexe avec des fonctionnalités supplémentaires.

\end{document}