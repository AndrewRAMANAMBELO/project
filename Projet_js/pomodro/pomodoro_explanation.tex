
\documentclass{article}
\usepackage[french]{babel}
\usepackage{listings}
\usepackage{xcolor}

\definecolor{codegray}{gray}{0.9}
\lstdefinestyle{customhtml}{
    backgroundcolor=\color{codegray},
    language=HTML,
    basicstyle=\ttfamily,
    frame=single
}

\lstdefinestyle{customjs}{
    backgroundcolor=\color{codegray},
    language=JavaScript,
    basicstyle=\ttfamily,
    frame=single
}

\lstdefinestyle{customcss}{
    backgroundcolor=\color{codegray},
    language=CSS,
    basicstyle=\ttfamily,
    frame=single
}

\title{Explication du Projet Pomodoro}
\author{}
\date{}

\begin{document}

\maketitle

\section{Introduction}
Ce projet est une application de minuterie Pomodoro développée en HTML, CSS et JavaScript. L'utilisateur peut définir une durée de travail et de repos et démarrer un cycle Pomodoro.

\section{Fichier HTML}
Le fichier \texttt{index.html} définit la structure de la page. Voici un extrait du code :

\begin{lstlisting}[style=customhtml]
<!DOCTYPE html>
<html lang="fr">
<head>
    <meta charset="UTF-8">
    <meta name="viewport" content="width=device-width, initial-scale=1.0">
    <title>Pomodoro</title>
    <link rel="stylesheet" href="style.css">
</head>
<body>
    <h1>Pomodoro</h1>
    <div class="time-inputs">
        <label for="workTime">Temps de travail (minutes):</label>
        <input type="number" id="workTime" value="25" min="1">
        <label for="restTime">Temps de repos (minutes):</label>
        <input type="number" id="restTime" value="5" min="1">
    </div>
    <script src="script.js"></script>
</body>
</html>
\end{lstlisting}

Ce fichier contient les entrées utilisateur pour définir les temps de travail et de repos.

\section{Fichier JavaScript}
Le fichier \texttt{script.js} gère la logique du compte à rebours. Voici un extrait du code :

\begin{lstlisting}[style=customjs]
const affichageTravail = document.querySelector('.affichageT');
const affichagePause = document.querySelector('.affichageP');
const btnGo = document.querySelector('.b1');
const btnPause = document.querySelector('.b2');
const btnReset = document.querySelector('.b3');
let tempsInitial;
let tempsDeRepos;
let pause = false;
let nbDeCycles = 0;
let ChronoEnMarche = false;
let timer;

function initializeTimes() {
    tempsInitial = workTimeInput.value * 60;
    tempsDeRepos = restTimeInput.value * 60;
}
btnGo.addEventListener('click', () => {
    if (!ChronoEnMarche) {
        initializeTimes();
        timer = setInterval(() => {
            if (!pause && tempsInitial > 0) {
                tempsInitial--;
            } else if (!pause && tempsInitial === 0 && tempsDeRepos > 0) {
                tempsDeRepos--;
            }
        }, 1000);
    }
});
\end{lstlisting}

Ce code met en place un intervalle pour réduire le temps de travail et de repos.

\section{Fichier CSS}
Le fichier \texttt{style.css} est responsable de l'apparence de la minuterie. Voici un extrait :

\begin{lstlisting}[style=customcss]
body {
    background: linear-gradient(to right, #333, #000022);
    font-family: Arial, sans-serif;
    color: #f1f1f1;
}
.bloc {
    width: 50%;
    display: flex;
    justify-content: center;
    align-items: center;
    flex-direction: column;
    font-size: 30px;
}
\end{lstlisting}

Ce fichier définit le style visuel de l'application.

\section{Conclusion}
Ce projet permet de gérer des sessions de travail et de repos à l'aide d'un minuteur interactif. Il combine HTML, CSS et JavaScript pour une expérience utilisateur fluide.

\end{document}
