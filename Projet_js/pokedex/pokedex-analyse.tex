\documentclass{article}
\usepackage[utf8]{inputenc}
\usepackage[french]{babel}
\usepackage{listings}
\usepackage{xcolor}
\usepackage{graphicx}
\usepackage{hyperref}

\definecolor{codegreen}{rgb}{0,0.6,0}
\definecolor{codegray}{rgb}{0.5,0.5,0.5}
\definecolor{codepurple}{rgb}{0.58,0,0.82}
\definecolor{backcolour}{rgb}{0.95,0.95,0.92}

\lstdefinestyle{mystyle}{
    backgroundcolor=\color{backcolour},   
    commentstyle=\color{codegreen},
    keywordstyle=\color{magenta},
    numberstyle=\tiny\color{codegray},
    stringstyle=\color{codepurple},
    basicstyle=\ttfamily\footnotesize,
    breakatwhitespace=false,         
    breaklines=true,                 
    captionpos=b,                    
    keepspaces=true,                 
    numbers=left,                    
    numbersep=5pt,                  
    showspaces=false,                
    showstringspaces=false,
    showtabs=false,                  
    tabsize=2
}

\lstset{style=mystyle}

\title{Analyse d'une Application Pokédex}
\author{Documentation Technique}
\date{\today}

\begin{document}

\maketitle

\section{Introduction}
Ce document présente une analyse détaillée d'une application web Pokédex qui utilise l'API PokeAPI pour afficher des informations sur les Pokémon. L'application est composée de trois fichiers principaux : un fichier JavaScript (script.js), un fichier HTML (index.html) et un fichier CSS (style.css).

\section{Structure de l'application}
L'application suit une architecture simple basée sur le DOM et les événements JavaScript. Elle permet aux utilisateurs de :
\begin{itemize}
    \item Visualiser une liste de Pokémon avec leurs noms, identifiants et images
    \item Rechercher des Pokémon par leur nom
    \item Charger plus de Pokémon en défilant vers le bas de la page
\end{itemize}

\section{Analyse du code JavaScript}

\subsection{Variables et sélecteurs}
Le script commence par sélectionner les éléments DOM nécessaires et initialiser les variables de contrôle :

\begin{lstlisting}[language=JavaScript]
const listPoke = document.querySelector('.list-poke');
const searchForm = document.querySelector('.recherche-poke');
const searchInput = document.querySelector('.recherche-poke input');
const loader = document.querySelector('.loader');

let allPokemon = [];
let displayedPokemon = [];
const limit = 1010; // On récupère TOUS les Pokémon disponibles
const initialDisplay = 24;
let index = initialDisplay;
const increment = 24;
\end{lstlisting}

Ces variables servent à :
\begin{itemize}
    \item Stocker les références aux éléments DOM
    \item Gérer les tableaux de données des Pokémon
    \item Contrôler le nombre de Pokémon affichés et le chargement progressif
\end{itemize}

\subsection{Récupération des données}
La fonction principale \texttt{fetchPokemonBase()} récupère la liste des Pokémon depuis l'API :

\begin{lstlisting}[language=JavaScript]
async function fetchPokemonBase() {
    loader.style.display = 'block';
    let url = `https://pokeapi.co/api/v2/pokemon?limit=${limit}`;
    let response = await fetch(url);
    let data = await response.json();
    let promises = data.results.map(pokemon => fetchPokemonComplet(pokemon));
    allPokemon = await Promise.all(promises);
    displayedPokemon = allPokemon.slice(0, initialDisplay);
    createCard(displayedPokemon);
    loader.style.display = 'none';
}
\end{lstlisting}

Cette fonction :
\begin{itemize}
    \item Affiche l'indicateur de chargement
    \item Effectue une requête à l'API PokeAPI
    \item Crée un tableau de promesses pour récupérer les détails de chaque Pokémon
    \item Attend la résolution de toutes les promesses avec \texttt{Promise.all()}
    \item Sélectionne les premiers Pokémon à afficher
    \item Masque l'indicateur de chargement une fois terminé
\end{itemize}

\subsection{Récupération des détails}
La fonction \texttt{fetchPokemonComplet()} récupère les détails spécifiques de chaque Pokémon :

\begin{lstlisting}[language=JavaScript]
async function fetchPokemonComplet(pokemon) {
    let response = await fetch(pokemon.url);
    let pokeData = await response.json();
    return {
        id: pokeData.id,
        name: pokeData.name,
        image: pokeData.sprites.front_default,
        types: pokeData.types.map(t => t.type.name)
    };
}
\end{lstlisting}

Cette fonction extrait et retourne uniquement les informations pertinentes :
\begin{itemize}
    \item Identifiant
    \item Nom
    \item URL de l'image
    \item Types du Pokémon
\end{itemize}

\subsection{Création des cartes}
La fonction \texttt{createCard()} génère les éléments HTML pour chaque Pokémon :

\begin{lstlisting}[language=JavaScript]
function createCard(pokemonArray) {
    pokemonArray.forEach(pokemon => {
        let li = document.createElement('li');
        li.classList.add('hoverableCarte');
        li.style.backgroundColor = getColorByType(pokemon.types[0]);
        
        li.innerHTML = `
            <h5>${pokemon.name}</h5>
            <p>#${pokemon.id}</p>
            <img src="${pokemon.image}" alt="${pokemon.name}">
        `;
        
        listPoke.appendChild(li);
    });
}
\end{lstlisting}

Cette fonction :
\begin{itemize}
    \item Parcourt le tableau de Pokémon
    \item Crée un élément de liste pour chaque Pokémon
    \item Définit la couleur de fond en fonction du type principal
    \item Ajoute le nom, l'identifiant et l'image du Pokémon
    \item Ajoute l'élément à la liste dans le DOM
\end{itemize}

\subsection{Gestion des couleurs}
La fonction \texttt{getColorByType()} associe une couleur à chaque type de Pokémon :

\begin{lstlisting}[language=JavaScript]
function getColorByType(type) {
    const colors = {
        fire: '#FDDFDF', water: '#DEF3FD', grass: '#DEFDE0', electric: '#FCF7DE', bug: '#F8D5A3',
        poison: '#E0A7F7', normal: '#F5F5F5', ground: '#F4E7DA', fairy: '#FCEAFF',
        fighting: '#E6E0D4', psychic: '#EAEDA1', rock: '#D5D5D4', ghost: '#705898',
        ice: '#96D9D6', dragon: '#97B3E6'
    };
    return colors[type] || '#F5F5F5';
}
\end{lstlisting}

Cette fonction utilise un objet de correspondance pour retourner une couleur hexadécimale en fonction du type de Pokémon.

\subsection{Gestion des événements}
Le script gère deux événements principaux :

\subsubsection{Recherche}
\begin{lstlisting}[language=JavaScript]
searchForm.addEventListener('submit', (e) => {
    e.preventDefault();
    let filter = searchInput.value.toLowerCase();
    let filteredPokemon = allPokemon.filter(p => p.name.includes(filter));
    listPoke.innerHTML = '';
    createCard(filteredPokemon);
});
\end{lstlisting}

Cet écouteur d'événements :
\begin{itemize}
    \item Empêche le comportement par défaut du formulaire
    \item Récupère la valeur saisie et la convertit en minuscules
    \item Filtre les Pokémon dont le nom contient la chaîne recherchée
    \item Vide la liste actuelle
    \item Affiche les résultats filtrés
\end{itemize}

\subsubsection{Défilement infini}
\begin{lstlisting}[language=JavaScript]
window.addEventListener('scroll', () => {
    if (window.innerHeight + window.scrollY >= document.body.offsetHeight - 50) {
        if (index < allPokemon.length) {
            let nextPokemon = allPokemon.slice(index, index + increment);
            index += increment;
            createCard(nextPokemon);
        }
    }
});
\end{lstlisting}

Cet écouteur d'événements :
\begin{itemize}
    \item Détecte quand l'utilisateur approche du bas de la page
    \item Vérifie s'il reste des Pokémon à afficher
    \item Sélectionne le prochain lot de Pokémon
    \item Met à jour l'index
    \item Ajoute les nouvelles cartes à la liste existante
\end{itemize}

\subsection{Initialisation}
Le script se termine par l'initialisation de l'application :

\begin{lstlisting}[language=JavaScript]
document.addEventListener('DOMContentLoaded', fetchPokemonBase);
\end{lstlisting}

Cette ligne déclenche la récupération des données dès que le DOM est entièrement chargé.

\section{Structure HTML}
Le fichier HTML fournit la structure de base de l'application :

\begin{lstlisting}[language=HTML]
<html lang="fr">
<head>
    <meta charset="UTF-8">
    <meta name="viewport" content="width=device-width, initial-scale=1.0">
    <title>Pokédex</title>
    <link rel="stylesheet" href="style.css">
</head>
<body>
    <header>
        <h1>Pokédex</h1>
        <form class="recherche-poke">
            <input type="text" placeholder="Rechercher un Pokémon...">
            <button type="submit">🔍</button>
        </form>
    </header>
    <ul class="list-poke"></ul>
    <div class="loader"></div>
    <script src="script.js"></script>
</body>
</html>
\end{lstlisting}

Les éléments clés sont :
\begin{itemize}
    \item Un en-tête avec le titre de l'application
    \item Un formulaire de recherche avec un champ de saisie et un bouton
    \item Une liste vide qui sera remplie par JavaScript
    \item Un indicateur de chargement
\end{itemize}

\section{Styles CSS}
Le fichier CSS définit l'apparence de l'application :

\begin{lstlisting}[language=CSS]
body {
    font-family: Arial, sans-serif;
    text-align: center;
    background-color: #f8f8f8;
    margin: 0;
    padding: 0;
}

header {
    background-color: #ff3e3e;
    padding: 20px;
    color: white;
}

.recherche-poke {
    margin-top: 10px;
}

.recherche-poke input {
    padding: 10px;
    width: 250px;
    border: 1px solid #ccc;
    border-radius: 5px;
}

.recherche-poke button {
    padding: 10px;
    border: none;
    background-color: white;
    cursor: pointer;
    border-radius: 5px;
}

.list-poke {
    display: flex;
    flex-wrap: wrap;
    justify-content: center;
    padding: 20px;
    list-style: none;
}

.hoverableCarte {
    background-color: white;
    padding: 15px;
    margin: 10px;
    border-radius: 10px;
    box-shadow: 0 4px 8px rgba(0, 0, 0, 0.2);
    text-align: center;
    transition: transform 0.3s;
}

.hoverableCarte:hover {
    transform: scale(1.05);
}

img {
    width: 100px;
    height: 100px;
}

.loader {
    width: 50px;
    height: 50px;
    border: 5px solid #f3f3f3;
    border-top: 5px solid #ff3e3e;
    border-radius: 50%;
    animation: spin 1s linear infinite;
    margin: 20px auto;
    display: none;
}

@keyframes spin {
    0% { transform: rotate(0deg); }
    100% { transform: rotate(360deg); }
}
\end{lstlisting}

Les styles principaux incluent :
\begin{itemize}
    \item Une mise en page centrée avec une police sans-serif
    \item Un en-tête rouge avec du texte blanc
    \item Des cartes avec des coins arrondis et des ombres
    \item Un effet d'agrandissement au survol des cartes
    \item Un indicateur de chargement animé
    \item Une disposition flexible pour les cartes qui s'adapte à différentes tailles d'écran
\end{itemize}

\section{Flux de l'application}
Le flux de l'application peut être résumé comme suit :
\begin{enumerate}
    \item Le DOM se charge et déclenche \texttt{fetchPokemonBase()}
    \item L'application récupère les données de tous les Pokémon
    \item Les 24 premiers Pokémon sont affichés
    \item L'utilisateur peut :
    \begin{itemize}
        \item Rechercher des Pokémon par nom
        \item Faire défiler pour charger plus de Pokémon
    \end{itemize}
\end{enumerate}

\section{Conclusion}
Cette application Pokédex est un exemple efficace d'une application web moderne qui utilise :
\begin{itemize}
    \item Des requêtes API asynchrones avec \texttt{async/await}
    \item La manipulation du DOM pour créer une interface dynamique
    \item Des événements utilisateur pour une expérience interactive
    \item CSS pour une présentation visuelle attrayante
    \item Une architecture simple mais fonctionnelle
\end{itemize}

L'application démontre comment créer une interface utilisateur réactive qui charge des données à la demande, offrant ainsi une expérience utilisateur fluide même avec un grand ensemble de données.

\end{document}